
%%%%%%%%%%%%%%%%%%%%%%% file typeinst.tex %%%%%%%%%%%%%%%%%%%%%%%%%
%
% This is the LaTeX source for the instructions to authors using
% the LaTeX document class 'llncs.cls' for contributions to
% the Lecture Notes in Computer Sciences series.
% http://www.springer.com/lncs       Springer Heidelberg 2006/05/04
%
% It may be used as a template for your own input - copy it
% to a new file with a new name and use it as the basis
% for your article.
%
% NB: the document class 'llncs' has its own and detailed documentation, see
% ftp://ftp.springer.de/data/pubftp/pub/tex/latex/llncs/latex2e/llncsdoc.pdf
%
%%%%%%%%%%%%%%%%%%%%%%%%%%%%%%%%%%%%%%%%%%%%%%%%%%%%%%%%%%%%%%%%%%%


\documentclass[runningheads,a4paper]{llncs}

\usepackage{amssymb}
\setcounter{tocdepth}{3}
\usepackage{graphicx}

\usepackage{url}
%\urldef{\mailsa}\path|{jmoura, andrea}@inf.ufsm.com.br|    
\newcommand{\keywords}[1]{\par\addvspace\baselineskip
\noindent\keywordname\enspace\ignorespaces#1}

\begin{document}

\mainmatter  % start of an individual contribution

% first the title is needed
\title{
Test automation on BPMS applications: \\ an Experience Report}

% a short form should be given in case it is too long for the running head
\titlerunning{
Test automation BPMS applications: \\ an Experience Report}

% the name(s) of the author(s) follow(s) next
%
% NB: Chinese authors should write their first names(s) in front of
% their surnames. This ensures that the names appear correctly in
% the running heads and the author index.
%
\author{Jessica Lasch de Moura
%\thanks{Please note that the LNCS Editorial assumes that all authors have used
%the western naming convention, with given names preceding surnames. This determines
%the structure of the names in the running heads and the author index.}%
\and Andrea Schwertner Charao}
%
\authorrunning{
Test automation BPMS applications: \\ an Experience Report}
% (feature abused for this document to repeat the title also on left hand pages)

% the affiliations are given next; don't give your e-mail address
% unless you accept that it will be published
\institute{Núcleo de Ciência da Computação\\
Universidade Federal de Santa Maria -- UFSM\\}
\mailsa\\
%\mailsb\\
%\mailsc\\
%\url{http://www.springer.com/lncs}}

%
% NB: a more complex sample for affiliations and the mapping to the
% corresponding authors can be found in the file "llncs.dem"
% (search for the string "\mainmatter" where a contribution starts).
% "llncs.dem" accompanies the document class "llncs.cls".
%

\toctitle{Lecture Notes in Computer Science}
\tocauthor{Authors' Instructions}
\maketitle


\begin{abstract}
%The abstract should summarize the contents of the paper and should contain at least 70 and at most 150 words. It should be written using the \emph{abstract} environment.
This article describes an experience of test automation of an application developed with the support of Business Process Management Systems - BPMS. For this purpose, we have implemented the same process using two different BPMS: Bonita and Activiti. We submit the resulting Web applications to two types of tests (load tests and functional tests), using test tools Apache JMeter, Selenium and Cucumber. The results show the feasibility and limitations of test automation for this type of application.

%\keywords{We would like to encourage you to list your keywords within the abstract section}
\end{abstract}


\section{Introduction}

%A gestão de processos de negócio (\emph{Business Process Management} -- BPM) tem suscitado o interesse de empresas e da comunidade científica, tanto por seus benefícios como por seus desafios. Designa-se por BPM o conjunto de conceitos, métodos e técnicas para suportar a modelagem, administração, configuração e análise de processos de negócio~\cite{weske}. Associados a isso, surgiram os sistemas BPM (\emph{Business Process Management Systems} -- BPMS), que são ferramentas de software para apoio ao ciclo de vida da gestão de processos de negócio. 

Business Process Management has aroused the interest of companies and the scientific community, both for their benefits as per their challenges. Is called of BPM a set of concepts, methods and techniques to support the modeling, administration, configuration and analysis of business processes  ~\cite{weske}. Associated with this, emerged the BPM systems (\emph{Business Process Management Systems} -- BPMS), which are software tools to support the life cycle of business process management.

%Dentre os diversos BPMS disponíveis atualmente, é comum encontrar ferramentas com suporte à modelagem, configuração e execução de processos de negócio. Em muitos casos, os BPMS abreviam o desenvolvimento de software, entregando aplicações Web completas para execução dos processos, usando tecnologias atuais e exigindo pouca escrita de código. Por outro lado, algumas tarefas como verificação e testes ainda são consideradas um desafio nesta área~\cite{aalst2013survey}. Em particular, o teste automatizado de aplicações de BPMS é pouco abordado, tanto pela comunidade da área de BPM~\cite{weske} como da área de testes de software~\cite{graham2012experiences}. 

Among the many BPMS currently available, it is common to find tools that support modeling, configuration and execution of business processes. In many cases, BPMS shorten the software development, delivering complete Web applications for process execution, using current technologies and requiring little code writing. On the other hand, some tasks such as checking and tests are still considered a challenge in this area ~\cite{aalst2013survey}. In particular, the automated test of BPMS applications is rarely addressed by both the BPM community ~\cite{weske} as software testing area~\cite{graham2012experiences.
 
%No entanto, a falta de automação nos testes pode levar a problemas durante a implementação e execução de processos de negócio, ainda mais quando se tratam de processos com muitas tarefas e fluxos de trabalho, que levam facilmente a explosões combinatórias.
However, the lack of automation in testing can lead to problems during implementation and execution of business processes, especially when treating processes with many tasks and workflows, which easily lead to combinatorial explosion.

%O propósito deste trabalho foi explorar soluções para teste automatizado de um processo implementado em BPMS. Para isso, partiu-se de uma aplicação real, em que testes manuais se revelaram insuficientes~\cite{sbsi2013}. No presente artigo, relata-se a experiência com testes automáticos nesta aplicação, utilizando-se ferramentas \emph{open source}. A aplicação é apresentada na seção \ref{s:apli}, após uma discussão sobre BPM e testes na seção \ref{s:bpmtest}. Na sequência, a seção \ref{s:testes} apresenta os métodos, ferramentas e resultados obtidos em cada tipo de teste. Por fim, a seção \ref{s:conclu} resume as lições aprendidas.

The purpose of this study was to explore solutions for automated testing of a process implemented in BPMS. For this, it started with a real application where manual testing proved inadequate~\cite{sbsi2013}. In this article, we report the experience with automated testing this application, using open source tools. The application is presented in section \ref{s:apli}, after a discussion of BPM and tests in section \ref{s:bpmtest}. Further , the section \ref{s:testes} presents the methods, tools and results of each type of test. Finally, the section \ref{s:conclu} summarizes the lessons learned.
%deixar citaçoes ao trabalho em protugues?


\section{BPM and Testing}\label{s:bpmtest}

%O termo BPM pode ser usado com ênfases diferentes, às vezes com foco em tecnologia (software) e outras vezes em gestão. Mesmo assim, a área tem convergido no entendimento do ciclo de vida de aplicações de BPM, que envolve as atividades de análise, modelagem, execução, monitoramento e otimização~\cite{ABPMP}. Também há convergência sobre o padrão BPMN (\emph{Business Process Model and Notation}) para expressar a modelagem de processos.

The BPM term can be used with different emphases, sometimes focusing on technology (software) and other times management. Still, the area has converged in understanding the lifecycle of BPM applications, involving the analysis of activities, modeling, execution, monitoring and optimization~\cite{ABPMP}. There is also convergence on the pattern BPMN (\emph{Business Process Model and Notation}) to express the process modeling. 
%citaçao em portugues?

%Os sistemas de BPM (BPMS) têm se afirmado como ferramentas essenciais para suporte a atividades desse ciclo de vida. Atualmente, pode-se dizer que um típico BPMS oferece recursos para definição e modelagem de processos em BPMN, controle da execução e monitoramento de atividades dos processos~\cite{forrester}. Há uma tendência dos BPMS em abreviar o desenvolvimento de software, por exemplo através de geradores de formulários Web associados a tarefas dos processos~\cite{greenresearch}. Nota-se, no entanto, que a preocupação com testes não fica evidente nas ferramentas BPMS. De fato, examinando-se o material promocional e a documentação disponível sobre os principais BPMS, observa-se uma ênfase em etapas de modelagem e execução.

BPM systems (BPMS) has been claimed as essential tools to support the activities of this life cycle. Currently, it can be said that a typical BPMS provides resources for definition and process modeling in BPMN, execution control and activity monitoring of processes~\cite{forrester}. There is a tendency of BPMS in shortening software development, for example via web forms generators associated with process tasks~\cite{greenresearch}. Note, however, that the concern with testing is not evident in the BPMS tools. Indeed, examining the promotional material and documentation available on the main BPMS, there is an emphasis on modeling and execution stages.


%Por outro lado, a importância dos testes é amplamente reconhecida em engenharia de software~\cite{swebok14}. 
%Considerando que aplicações de BPMS são geralmente sistemas baseados na Web, pode-se supor que sejam testadas com sucesso usando-se abordagens consagradas, como por exemplo testes de carga ou testes funcionais do tipo caixa-preta. Há também quem argumente que o teste de aplicações de BPM difira do teste de aplicações Web tradicionais~\cite{evoke}, porém não foram encontradas mais referências aprofundando esse ponto de vista. Essa constatação reforçou a motivação para o presente trabalho.
Moreover, the importance of the tests is widely recognized in software engineering ~\cite{swebok14}. 
Whereas BPMS applications are usually Web-based systems, it can be assumed that it can be successfully tested using dedicated approaches, such as load tests or functional testing of the black-box type. There are also those who argue that BPM applications test differs from traditional Web applications testing~\cite{evoke}, but there were no more references to deepen this view. This finding reinforced the motivation for this work. 


\section{Test Target Application}\label{s:apli}

%A aplicação alvo deste trabalho refere-se a um processo comum em instituições de ensino superior: a apreciação de Atividades Complementares de Graduação (ACGs), ou seja, atividades que formam a parte flexível do currículo de graduandos (participação em palestras, eventos, projetos, etc.). Em um trabalho anterior ~\cite{sbsi2013}, apresentou-se a modelagem, implementação e implantação desse processo. Sua representação em BPMN, na Figura \ref{fig:diagrama}, revela um total de 11 tarefas distribuídas em 5 divisões de responsabilidade (Aluno, Tutor, etc.).

The target application of this work refers to a common process in higher education institutions: the appreciation of complementary activities, ie, activities that form the flexible part of the undergraduate curriculum (participation in lectures, events, projects , etc.). In a previous work ~\cite{sbsi2013}, presented the modeling, implementation and deployment of this process. Their representation in BPMN, in Figure \ref{fig:Diagram}, reveals a total of 11 tasks span responsibility 5 divisions.
%citando trabalho em portugues

In previous work, we implemented the process with the Bonita BPM tool \footnote{Bonita BPM (formerly Bonita Open Solution). Available at:. Www.bonitasoft.com}, an open source BPMS recognized in the corporate world ~\cite{forrester}. The resulting application has several Web forms for each division of responsibility, being the first one for the data population by the student. According to the type of complementary activities, the flow is directed to those responsible for the assessment and validation of activity. Note, in Figure \ref{fig:Diagram}, the process has several gateways, which leads to over 15 possible paths in the process.

\begin{figure}[ht]
\centering
\includegraphics[max size={14cm}{14cm}]{figuras/processo.png}
\caption{Diagrama do processo em BPMN}
\label{fig:diagrama}
\end{figure}

%A aplicação foi submetida a testes funcionais realizados manualmente, além de testes de aceitação realizados com um grupo de usuários reais. Entretanto, com algumas semanas em operação, surgiram problemas: instâncias do processo falharam devido a entradas inesperadas, serviços não foram restabelecidos corretamente após serem interrompidos e houve sobrecarga devido ao grande número de casos abertos numa data limite. Essa experiência motivou a busca de soluções para automação de testes.
The application was subjected to functional testing performed manually, in addition to acceptance testing with a group of real users. However, a few weeks in operation, problems arose: Process instances failed due to unexpected inputs, services were not restored properly after being interrupted and there was overloaded because of the number of cases opened on a deadline. This experience led to the search for solutions for test automation.


%Verificou-se, no entanto, que o BPMS utilizado não possuía suporte a nenhum tipo de teste automatizado. Buscou-se outros BPMS com licenças \emph{open source} ou \emph{freeware}, que pudessem implementar o processo em questão e que oferecessem suporte a testes. A preferência por este tipo de licença foi devida à sua flexibilidade e viabilidade financeira para o ``cliente'' deste projeto: uma instituição pública de ensino.
It was found, however, that the BPMS used does not support any type of automated testing. We attempted to other BPMS with open source or freeware licenses, which could implement the process in question and offered testing support. The preference for this type of license was due to its flexibility and affordability to the "client" this project: a public educational institution.
%Dentre as ferramentas analisadas (TIBCO\footnote{TIBCO. Disponível em: www.tibco.com.}, Activiti\footnote{Activiti. Disponível em: www.activiti.org.}, Process Maker\footnote{Process Maker. Disponível em: www.processmaker.com.}, Intalio\footnote{Intalio. Disponível em: www.intalio.com.}), nenhuma oferecia evidente suporte a testes automatizados. Apenas a ferramenta Activiti citava a possibilidade de testes de unidade aliados ao JUnit, porém sem muitas informações sobre essa alternativa. Por isso partiu-se para outra opção: utilizar ferramentas de teste externas ao BPMS. 
Among the analyzed tools (TIBCO\footnote{TIBCO Available at:.. Www.tibco.com}, Activiti\footnote{Activiti Available at:.. Www.activiti.org}, Process Maker\footnote{Process Maker. Available in: .. www.processmaker.com}, Intalio\footnote{Intalio Available at:. www.intalio.com}), offered no clear support for automated testing. Just Activiti tool cited the possibility of allies unit tests to JUnit, but not much information about this alternative. So we went to another option: use test tools external to BPMS.

%Adicionalmente, decidiu-se implementar a mesma aplicação usando outro BPMS, a fim de ampliar a experiência e, possivelmente, identificar semelhanças e diferenças no teste automatizado de implementações com diferentes BPMS. Porém, não tinha-se como objetivo eleger o melhor BPMS, mas sim avaliar suporte aos testes considerados. 
In addition, it was decided to deploy the same application using other BPMS in order to broaden the experience and possibly identify similarities and differences in automated test implementations with different BPMS. Yet it had intended to elect the best BPMS, but to assess support for both tests.

%A ferramenta escolhida foi Activiti, um BPMS baseado em tecnologias Java, assim como Bonita, permitindo trabalhar com as mesmas tecnologias do lado servidor (Tomcat e MySQL, no caso em questão). Esse BPMS também trabalha com a mesma versão da notação BPMN usada no Bonita, permitindo importar na íntegra o processo originalmente criado. Os formulários Web criados com ajuda do Bonita, no entanto, não puderam ser importados e tiveram de ser recriados com Activiti.
The tool chosen was Activiti, a BPMS based on Java technologies, like Bonita, allowing you to work with the same server-side technologies (Tomcat and MySQL, in this case). This BPMS also works with the same version of the BPMN notation used in Bonita, allowing import in full the process originally created. Web forms created with Bonita's help, however, could not be imported and had to be recreated with Activiti.

\section{Description and Performance of tests}\label{s:testes}

%No planejamento de testes automatizados, priorizou-se o teste de aspectos que de fato revelaram problemas durante a implantação, no trabalho precedente. Os testes escolhidos foram: (a) testes de carga, que são um tipo de teste de desempenho, visando avaliar o comportamento do sistema frente a um grande número de solicitações e (b) testes funcionais, a fim de verificar as saídas do sistema produzidas a partir de entradas pré-definidas. Nenhum destes tipos de teste possui suporte nos BPMS Bonita e Activiti, que incluem somente funcionalidades limitadas de simulação e depuração de execução dos processos. Embora o Activiti citasse suporte a testes de unidade, não explorou-se esta opção por entender-se que seria menos prioritária frente aos problemas observados.
In planning automated tests, priority was given to the test of aspects that in fact revealed problems during deployment, in previous work. The selected tests were: (a) load tests, which are one type of performance testing, to evaluate the opposite behavior of the system to a large number of requests, and (b) functional tests to verify the produced system output from pre-defined inputs. None of these types of tests is supported in BPMS Bonita and Activiti, which include only limited functionality simulation and debugging process execution. Although Activiti quoting support unit tests not be exploited for this option to be understood that priority would be less compared to the observed problems.
%Assim, realizou-se um levantamento de ferramentas destinadas ao teste de aplicações Web e selecionou-se as julgadas mais promissoras, antes de partir-se para o detalhamento e execução dos testes. Os \emph{scripts} que configuram os testes estão disponíveis para consulta em http://www.inf.ufsm.br/~andrea/bpmtest-scripts.zip.
Thus, there was a survey of tools intended for Web application testing and was selected the most promising judged before departing to the detailing and execution of tests. The scripts that configure the tests are available for consultation in http://www.inf.ufsm.br/~andrea/bpmtest-scripts.zip.

\subsection{Load Tests}

%Testes de carga em aplicações Web são tipicamente realizados gerando-se múltiplas requisições HTTP ao servidor, de forma controlada. Para isso, uma etapa crítica é a identificação das requisições que devem ser reproduzidas. Existem diversas ferramentas que se propõem a facilitar este tipo de teste, dentre as quais pode-se citar: JMeter\footnote{Apache JMeter. Disponível em: www.jmeter.apache.org.}, The Grinder\footnote{The Grinder. Disponível em: www.grinder.sourceforge.net/.} e WebLOAD\footnote{WebLOAD. Disponível em: radview.com/webload-download/.}. Todas são ferramentas \emph{open source} e possuem diversas opções, mas escolheu-se a ferramenta JMeter pela sua funcionalidade ``proxy server'' que é de grande auxílio à captura das requisições.
Load tests on web applications is typically performed by generating multiple HTTP requests to the server in a controlled manner. For this, a critical step is to identify the requests that should be played. There are several tools that purport to facilitate this type of testing, among which we can mention: JMeter\footnote{Apache JMeter. Available at:. www.jmeter.apache.org}, The Grinder\footnote{The Grinder. Available at:. www.grinder.sourceforge.net/} and WebLOAD\footnote{WebLOAD. Available at:. Radview.com/webload-download/}. All tools are open source and have several options, but chose to JMeter tool for its functionality "proxy server" which is a great help to the capture of requests.

%Um teste da aplicação com JMeter consiste em quatro etapas: capturar as requisições HTTP, exportar as requisições (formato .jrxml para JMeter), configurar o plano de teste e, por fim, executar o teste no JMeter. A aplicação gera diferentes requisições HTTP que precisam ser identificadas e interpretadas, para serem reproduzidas automaticamente. O emprego de tecnologias Web com processamento assíncrono, do lado do cliente, pode dificultar esta etapa, pois uma ação do usuário pode não gerar imediatamente uma requisição ao servidor. Além disso, em aplicações de BPMS, diversos usuários atuam sobre diferentes tarefas de um mesmo processo, de modo que as requisições HTTP carregam chaves identificando usuários e processos.
An application test with JMeter consists of four steps: capture HTTP requests, export requests (.jrxml format for JMeter), set up the test plan and, finally, run the test in JMeter. The application generates different HTTP requests that need to be identified and interpreted to be played automatically. The use of Web technologies with asynchronous processing, client-side, can hinder this step because a user action can not immediately generate a request to the server. Moreover, in applications BPMS, many users act on different tasks in a same process, so that the HTTP requests that carry user identifying keys and processes.


%No caso da aplicação criada com Bonita, verificou-se que existe uma chave identificadora de sessão que é gerada no momento em que usuário acessa o sistema e outra chave identificadora de instância, ou seja, que identifica cada execução do processo como única, sendo criada pelo servidor no momento em que o usuário inicia o processo. Para executar os testes, portanto, foi necessário localizar a requisição em que essas chaves são geradas e utilizar a ferramenta ``Extrator de Expressão Regular'' do JMeter para obter seus valores. Já na aplicação criada com Activiti, foi inviável identificar a requisição em que as chaves são geradas, pois não há uma requisição cujo retorno (resposta do servidor) contenha valores de chaves. Esta situação leva a crer que a geração das chaves identificadoras é feita internamente pelo BPMS, ou seja, não em uma requisição HTTP e, por consequência, esta não pode ser capturada e reproduzida no JMeter.
In the case of application created with Bonita, it was found that there is a session identifier key that is generated at the time user accesses the system and other instance identifying key, that is, identifies each execution of the process as only being created by the server when the user initiates the process. To run the tests, so it was necessary to locate the request in which these keys are generated and use the "Regular Expression Extractor" tool of JMeter to get their values. In the application created with Activiti, it was impossible to identify the request in which the keys are generated, as there is not a request whose return (server response) contains key values. This suggests that the generation of the identifying key is made internally by the BPMS, ie not in an HTTP request and therefore this can not be captured and reproduced in JMeter.

%parei a traduçao aqui
\subsubsection{Resultados dos Testes de Carga}


Devido aos problemas relatados na seção anterior, os testes de carga só puderam ser realizados na aplicação criada com Bonita. A fim de testar o comportamento do sistema com diferentes níveis de carga, foram executados testes com 1, 50, 100 e 200 usuários virtuais e foram analisadas requisições referentes a etapas essenciais para iniciar o processo: efetuar login, exibir a página inicial do Bonita, selecionar o processo, exibir o formulário inicial (Aluno solicita ACG) e enviar o formulário preenchido. A aplicação foi hospedada em um servidor SGI Altix XE 210 com 24 GB de RAM, acessível por uma rede Fast Ethernet.

Devido aos problemas relatados na seção anterior, os testes de carga só puderam ser realizados com a aplicação criada com Bonita. A fim de testar o comportamento do sistema com diferentes níveis de carga, foram executados testes com 1, 50, 100 e 200 usuários virtuais/acessos simultâneos e foram analisadas as requisições referentes às etapas: efetuar login, exibir a página inicial do Bonita, selecionar o processo, exibir o formulário inicial (Aluno solicita ACG) e enviar o formulário preenchido. Os testes foram executados em um servidor com 24 GB de RAM e 2 processadores Intel Xeon E5520, com 4 núcleos. 



Os tempos de resposta de cada etapa, em função do número de usuários, podem ser vistos na Tabela \ref{tab:resultadoCarga}. Os resultados mais alarmantes são para 200 usuários virtuais, em que o tempo médio de resposta na requisição de login foi de 10.149 ms, ou seja, aproximadamente 10 s, o que é um tempo de resposta alto. A média de tempo de resposta para todas requisições foi de 3.111 ms (ou seja, 3 s), e o desvio padrão foi de 13.088. Além dos altos tempos de resposta, o teste com 200 usuários virtuais apresentou taxas de erro, em algumas requisições, que não foram encontradas com um número menor de usuários. Por exemplo, a requisição que executou login apresentou uma taxa de 2\% de erro e, ao todo, as requisições obtiveram uma taxa de erro de 7.82\%. Os erros ocorreram devido a requisições sem resposta. %Confrontando tais erros com logs do Bonita, estes indicam timeouts internos em alguns métodos, mas os dados não permitiram identificar precisamente a origem dos erros.


Os tempos de resposta de cada etapa, em função do número de usuários, podem ser vistos na Tabela \ref{tab:resultadoCarga}. Os resultados mais alarmantes são para 200 usuários virtuais, em que o tempo médio de resposta na requisição de login foi de 10.149 ms, ou seja, aproximadamente 10 s, o que é um tempo de resposta alto. Além dos altos tempos de resposta, o teste com 200 usuários virtuais apresentou taxas de erro, em algumas requisições, que não foram encontradas com um número menor de usuários. Por exemplo, a requisição que executou login apresentou uma taxa de 2\% de erro e, ao todo, as requisições obtiveram uma taxa de erro de 7.82\%.

\begin{table}
{\scriptsize
\centering
\begin{tabular}{p{2cm}|p{2cm}|p{2cm}|p{2cm}|p{2cm}|p{2cm}}
\hline
Usuários & Login & Pág. Inicial & Seleção Processo & Form. Inicial & Enviar form. \\\hline
1 & 126 & 32 & 38 & 80 & 73\\\hline
50 & 597 & 191 & 179 & 368 & 152\\\hline
100 & 1972 & 571 & 552 & 760 & 694\\\hline
200 & 10.149 & 3.239 & 934 & 2.122 & 1.918\\\hline
\end{tabular}
}
\caption{Tempos médios de resposta, em milissegundos}
\label{tab:resultadoCarga}
\end{table}


De forma geral, portanto, o teste de carga com JMeter atingiu seus objetivos e ajuda a explicar a sobrecarga que ocorreu com a aplicação em produção, quando muitos alunos tentaram acessar o formulário inicial numa data limite. No entanto, é importante ressaltar que este teste foi executado apenas em etapas iniciais do processo e, mesmo assim, já foi trabalhoso e consumiu algumas horas de preparação, por exigir uma análise profunda das requisições HTTP para executar os testes com sucesso. Ao todo, foram capturadas cerca de 100 requisições só nestas etapas, portanto estima-se que o teste de uma tarefa mais ao final do processo possa se tornar inviável com JMeter, por demandar a identificação e interpretação de muitas requisições. Outra observação importante, nesta experiência, é que esta abordagem de teste sofre com a dependência das tecnologias Web empregadas pelo BPMS.

\subsection{Testes Funcionais}

Para executar testes funcionais em aplicações Web, pode-se utilizar ferramentas livres como Selenium\footnote{Selenium. Disponível em: www.seleniumhq.org.}, Watir\footnote{Watir. Disponível em: www.watir.com.} ou Geb\footnote{Geb. Disponível em: www.gebish.org.}. Para este trabalho, escolheu-se a ferramenta Selenium, aliada ao Cucumber-JVM\footnote{Cucumber-JVM. Disponível em: www.github.com/cucumber/cucumber-jvm.} para descrição dos testes. A escolha foi motivada pelo grande número de referências ao Selenium na Web, confirmadas por um trabalho que apresentou resultados satisfatórios com Selenium e Cucumber~\cite{sbqs2013}.

Com estas ferramentas, o processo para a execução de um teste funcional é composto de: captura da interação do usuário com o navegador (Selenium IDE), exportação do código gerado (Selenium IDE), criação do cenário de teste (Cucumber), criação das definições dos passos do teste (Cucumber), criação dos métodos para cada passo (Java) e execução do teste (Selenium WebDriver). O cenário de teste é a definição, em ordem de execução, dos passos que são executados na aplicação, bem como dos resultados esperados. Para este trabalho, definiu-se um cenário em que o aluno faz login e preenche 2 formulários referentes à primeira tarefa do processo (Aluno solicita ACG). Para este cenário, foram criados métodos variando as entradas nos formulários. Como o processo testado é o mesmo em ambos os BPMS, o cenário de teste também é o mesmo. 


Para as aplicações de ambos os BPMS, inicialmente ocorreram erros na execução dos testes, relativos à localização de campos nos formulários Web. Verificou-se que os campos estavam localizados em um \emph{iframe} e, embora a captura da interação ocorresse sem problemas, o código gerado não selecionava o \emph{iframe} e por isso não encontrava os campos. Assim, foi necessário utilizar um método do Selenium para acessar o \emph{iframe} antes de selecionar o elemento desejado.


Na aplicação com Activiti, ocorreu também outro problema. Diferente do que ocorreu com os formulários Web gerados pelo Bonita, o Selenium IDE não capturou toda a interação do usuário com a aplicação. De fato, na etapa de login, o Selenium capturou apenas o acesso à página e o ``clique'' ao botão de login, ou seja, não capturou o preenchimento dos campos ``Usuário'' e ``Senha''. Este problema se repetiu com alguns outros elementos do formulário Web durante a gravação da interação. Acredita-se que o problema ocorra devido à estrutura das página Web, que pode conter elementos que o Selenium não identifique automaticamente, tais como \emph{divs}, \emph{frames} e \emph{scripts}, por exemplo. No entanto, isso não impossibilita a criação e execução dos testes. Para contornar o problema, foi necessário estudar a estrutura das páginas Web em questão, localizar os elementos faltantes e então adicionar o código para acessá-los nos respectivos métodos.


\subsubsection{Resultados dos Testes Funcionais}

Os testes funcionais mostraram-se mais viáveis do que os testes de carga, por não exigir análise (trabalhosa) das interações com o servidor. Pode-se dizer que este teste atingiu todos seus objetivos, pois permitiu reproduzir a interação do usuário, bem como criar o código para testar as aplicações com diferentes entradas, num cenário envolvendo a tarefa inicial do processo.

Os testes funcionais mostraram-se mais viáveis do que os testes de carga, pois uma boa parcela da interação é executada no lado cliente, sem necessidade de lidar explicitamente das interações com o servidor. Pode-se dizer que este teste atingiu todos seus objetivos, pois permitiu reproduzir a interação do usuário, bem como criar o código para testar as aplicações com diferentes entradas, num cenário envolvendo a tarefa inicial do processo, ampliando a cobertura dos testes. O  teste funcional também reproduziu erros encontrados em produção e que tinham passado despercebidos nos testes manuais (entradas não previstas corretamente nos formulários), esses erros foram os mesmos em ambas as ferramentas BPMS, pois os formulários foram configurados de modo equivalente. 

O teste funcional também foi menos dependente do BPMS. Na Tabela \ref{tab:testeFuncional} apresenta-se um resumo das principais semelhanças e diferenças encontradas. Em ambos os casos, foram necessárias poucas modificações no código gerado pelo Selenium e Cucumber, bastando para isso inspecionar a estrutura das páginas Web. O Cucumber torna a implementação dos testes mais rápida e menos trabalhosa do que se fosse usado apenas o Selenium, abreviando a geração de código alinhado com os cenários de teste. Mesmo assim, caso seja necessário estender os testes a muitas tarefas de um processo, as modificações no código de teste podem se tornar trabalhosas.


\begin{table}
\begin{center}
{\scriptsize
\begin{tabular}{p{6cm}|l|p{4.5cm}}
\hline
 & Bonita & Activiti \\\hline
Tecnologias Web & HTML, CSS, Ajax & HTML, CSS, Ajax \\\hline
Captura da interação do usuário utilizando o Selenium & Total & Parcial (necessitou de inser\-ção manual de alguns campos) \\\hline
Foi possível exportar o código gerado pelo Selenium? & Sim & Sim \\\hline
Reconhecimento de todos os campos capturados SEM alteração de código & Parcial & Parcial \\\hline
Reconhecimento de todos os campos capturados COM alteração de código & Total & Total \\\hline
\end{tabular}
}
\caption{Resumo comparativo sobre o teste funcional}
\label{tab:testeFuncional}
\end{center}
\end{table}


\section{Considerações Finais}\label{s:conclu}
Neste trabalho, explorou-se soluções de teste automatizado em uma aplicação de BPMS. Na ausência de suporte a testes de carga e funcionais nos BPMS Bonita e Activiti, aplicou-se ferramentas de teste voltadas a aplicações Web em geral. 

%Pode-se dizer que esse objetivo foi atingido pois, durante a execução do trabalho, foi possível obter várias conclusões sobre o teste automatizado deste tipo de software.

%Uma das abordagens adotadas foi de que, por aplicações BPM serem aplicações WEB, poderiam ser tratadas e testadas como um software em geral, utilizando ferramentas consagradas para tal. De fato, esta abordagem mostrou algumas desvantagens e dificuldades relacionadas a particularidades das aplicações BPM.

No que concerne ao teste de carga, este mostrou-se útil para explicar falhas observadas no trabalho anterior. Também mostrou-se um teste trabalhoso, ou até inviável, dependendo do BPMS usado. A experiência com dois BPMS fortaleceu essa conclusão, pois ocorreram situações distintas: com Bonita o teste foi bem sucedido, porém com Activiti o teste não pôde ser executado, já que não se conseguiu reproduzir todas requisições. 
%Assim, é prudente O BPMS deve ser avaliado antes da execução do teste para verificar se este é viável ou não.

No teste funcional, com a abordagem adotada, obteve-se maior sucesso na execução dos testes e observou-se uma menor dependência dos BPMS, em comparação com o teste anterior. A tarefa de teste pode vir a ser trabalhosa, principalmente quando deseja-se testar muitas tarefas e fluxos que um processo de negócio pode ter.
%\begin{figure}[ht]
%\centering
%\includegraphics[width=.99\textwidth]{figuras/tabelaCaminhos.png}
%\caption{Possíveis caminhos}
%\label{fig:tabelaCaminhos}
%\end{figure}
%Nas ferramentas abordadas no neste trabalho ou nas ferramentas estudadas e tabeladas, não foi encontrado nenhum tipo de suporte a teste automatizado dos processos, uma particularidade dos processos é que estes podem possuir diversos caminhos e estes caminhos precisam ser testados. Na Tabela \ref{tab:tabelaCaminhos} foram exibidos cujos testes foram executados manualmente e, com base no estudo executado neste trabalho e nas ferramentas estudadas, pode-se afirmar que atualmente é muito difícil encontrar um BPMS com suporte a este tipo de teste.

De modo geral, a experiência mostrou que, sob certas condições, é viável testar aplicações de BPMS com ferramentas de teste para sistemas Web. A principal condição, no caso considerado, foi o direcionamento dos testes a uma tarefa inicial do processo, identificada como gargalo. Como o tempo e esforço para uma única tarefa foi significativo, essa abordagem pode se tornar inviável caso seja necessário aumentar a cobertura dos testes.

De modo geral, como lições aprendidas temos que, sob certas condições, é viável testar aplicações de BPMS com ferramentas de teste para sistemas Web. A principal condição, no caso considerado, foi o direcionamento dos testes a uma tarefa inicial do processo, identificada como gargalo. Como o tempo e esforço para uma única tarefa foi significativo, essa abordagem pode se tornar inviável caso seja necessário estender os testes a muitas tarefas de um processo.


Outro aspecto a ser considerado é que, quando o BPMS implementa a interação com o usuário e/ou com servidores, o desenvolvedor deixa de escolher e controlar todas as tecnologias utilizadas. Essa facilidade, no entanto, pode dificultar a automação de testes com ferramentas externas, que se beneficiam de conhecimento sobre a implementação (por exemplo, uso de \emph{iframes}, Ajax, na experiência em questão).

Embora a experiência deste trabalho não tenha sido exaustiva, pôde-se notar que o suporte a testes automatizados é pouco explorado em BPMS. Entende-se que esta seria uma funcionalidade bem-vinda, supondo-se que abreviaria uma etapa essencial para garantir a qualidade do software resultante. 



\begin{thebibliography}{4}

\bibitem{ABPMP} Guide to the Business Process Management Body of Knowledge (BPM
  CBOK). Association of Business Process Management Professionals, 2nd
  edition (2012)

\bibitem{swebok14}
Bourque, P. e Fairley, R.~E.: Guide to the Software Engineering Body of Knowledge (SWEBOK), Version 3.0. IEEE Computer Society (2014)

\bibitem{evoke} Chetty, N.~K.: How to perform workflow testing for BPM applications (2014), \url{http://www.evoketechnologies.com}

\bibitem{sbqs2013}
Chiavegatto, R. et~al.: Especificação e automação colaborativas de testes utilizando a técnica {BDD}. In {\em XII Simpósio Brasileiro de Qualidade de Software}, pp.
  334--341, (2013)



\bibitem{sbsi2013}
de~Moura, J.~L. et~al.: Gestão de processos de negócio em curso de sistemas de
  informação: Relato de experiência utilizando software livre. In {\em IX Simpósio Brasileiro de Sistemas de Informação}, pp.
  206--217, (2013)

\bibitem{forrester}
Forrester Research.: The forrester wave: {BPM} suites, {Q1} (2013)

\bibitem{graham2012experiences}
Graham, D. e Fewster, M.: Experiences of Test Automation: Case Studies of Software Test Automation. Addison-Wesley (2012)

\bibitem{myers}
Myers, G.~J. e Sandler, C.: The Art of Software Testing. John Wiley \& Sons, (2011)

\bibitem{aalst2013survey}
van~der Aalst, W. M.~P.: Business process management: A comprehensive survey. {\em ISRN Software Engineering}, 2013(507984)

\bibitem{weske}
Weske, M.: Business Process Management: Concepts, Languages, Architectures. Springer, 2nd edition, (2012)

\bibitem{greenresearch}
Winter Green Research.: Business process management (BPM) cloud, mobile, and patterns: Market shares, strategies, and forecasts, worldwide, 2013 to 2019.

\end{thebibliography}


\end{document}

